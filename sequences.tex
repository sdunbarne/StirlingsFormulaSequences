%%% -*-LaTeX-*-
%%% sequences_new.tex.orig
%%% Prettyprinted by texpretty lex version 0.02 [21-May-2001]
%%% on Thu Jan 21 08:57:08 2021
%%% for Steven R. Dunbar (sdunbar@family-desktop)

\documentclass[12pt]{article}

\input{../../../../etc/macros}
\input{../../../../etc/mzlatex_macros}
%% \input{../../../../etc/pdf_macros}

\bibliographystyle{plain}

\begin{document}

\myheader \mytitle

\hr

\sectiontitle{Stirling's Formula from Simple Functions, Sequences and
Series}

\hr

\usefirefox

\hr

\visual{Rating}{../../../../CommonInformation/Lessons/rating.png}
\section*{Rating} %one of
% Everyone: contains no mathematics.
% Student: contains scenes of mild algebra or calculus that may require guidance.
% Mathematically Mature: may contain mathematics beyond calculus with proofs.
Mathematicians Only:  prolonged scenes of intense rigor.

\hr

\visual{Section Starter Question}{../../../../CommonInformation/Lessons/question_mark.png}
\section*{Section Starter Question} What is the geometric summation
formula?  How can you use the geometric sum formula to derive the series
expansion for \( \log( 1 + x) \)\,?  What do you need to know about the
geometric sum formula to justify its use to derive the series expansion
for \( \log( 1 + x) \)\,?

\hr

\visual{Key Concepts}{../../../../CommonInformation/Lessons/keyconcepts.png}
\section*{Key Concepts}

\begin{enumerate}
    \item
        \defn{Stirling's Formula}, also known as Stirling's Approximation,
        is the asymptotic relation
        \[
            n!  \asympt \sqrt{2 \PI} n^{n+1/2} \EulerE^{-n}\,.
        \]
    \item
        The formula is useful in estimating large factorial values, but
        its main mathematical value is in limits involving factorials.
    \item
        An improved inequality version of Stirling's Formula is
        \[
            \sqrt{2 \PI} n^{n+1/2} \EulerE^{-n + 1/(12n+1)} < n!  <
            \sqrt{2 \PI} n^{n+1/2} \EulerE^{-n + 1/(12n)}\,.
        \]
    \item
        Some related inequalities and asymptotics for binomial
        coefficients are
        \[
            \left( \frac{n}{k} \right)^k \le \binom{n}{k} \le \left(
            \frac{\EulerE n}{k} \right)^k
        \] and
        \[
            \binom{n}{k} \asympt \left(\frac{n^k}{k!} \right)
        \] if \( k = o(n^{1/2}) \) as \( n \to \infty \).
\end{enumerate}

\hr

\visual{Vocabulary}{../../../../CommonInformation/Lessons/vocabulary.png}
\section*{Vocabulary}
\begin{enumerate}
    \item
        \defn{Stirling's Formula}, also known as Stirling's Approximation,
        is the asymptotic relation
        \[
            n!  \asympt \sqrt{2 \PI} n^{n+1/2} \EulerE^{-n}\,.
        \]
    \item
        The \defn{double factorial} notation is \( n!!  = n \cdot (n-2)
        \cdots 4 \cdot 2 \) if \( n \) is even, and \( n!!  = n \cdot (n-2)
        \cdots 3 \cdot 1 \) if \( n \) is odd.
\end{enumerate}

\hr

\visual{Mathematical Ideas}{../../../../CommonInformation/Lessons/mathematicalideas.png}
\section*{Mathematical Ideas}

Following usual mathematical conventions in subjects beyond calculus,
all logarithms are natural logarithms with base \( \EulerE \).

\defn{Stirling's Formula}, also known as Stirling's Approximation, is the
asymptotic relation%
\index{Stirling's Formula}
\[
    n!  \asympt \sqrt{2 \PI} n^{n+1/2} \EulerE^{-n}\,.
\] The formula is useful in estimating large factorial values, but its
main mathematical value is in limits involving factorials.  Another
attractive form of Stirling's Formula is:
\[
    n!  \asympt \sqrt{2 \PI n } \left( \frac{n}{\EulerE} \right)^n\,.
\]

An improved inequality version of Stirling's Formula is
\begin{equation}
    \sqrt{2 \PI} n^{n+1/2} \EulerE^{-n + 1/(12n+1)} < n!  < \sqrt{2 \PI}
    n^{n+1/2} \EulerE^{-n + 1/(12n)}\,.%
    \label{eq:sequences:stirlingsinequality}
\end{equation}
\index{Stirling's Formula!inequality version}
See \link{http://mathworld.wolfram.com/StirlingsApproximation.html}{Stirling's
Formula} in MathWorld.com.

A consequence of the improved inequality is the simple and useful
inequality about Stirling's Formula for all \( n \)
\[
    \sqrt{2 \PI} n^{n+1/2} \EulerE^{-n} < n!\,.
\]%
\index{Stirling's Formula!simple inequality version}

Here we rigorously derive Stirling's Formula using elementary sequences
and series expansions of the logarithm function, based on the sketch in
Kazarinoff
\cite{kazarinoff61}, based on the note by Nanjundiah
\cite{nanjundiah59}, using motivation and background from the article by
Hammett
\cite{hammett20}.

\subsection*{A weak form of Stirling's Formula}

\begin{theorem}
    \[
        \sqrt[n]{n!} \asympt \frac{n}{\EulerE}.
    \]%
    \label{thm:sequences:weakform}
\end{theorem}
\index{Stirling's Formula!weak asymptotic version}

\begin{remark}
    This form of Stirling's Formula is weaker than the usual form since
    it does not give direct estimates on \( n! \).  On the other hand,it
    avoids the determination of the asymptotic constant \( \sqrt{2\PI} \)
    which usually requires Wallis's Formula or equivalent.  For many
    purposes of estimation or limit taking this version of Stirling's
    Formula is enough, and the proof is elementary.  The proof is taken
    from
    \cite[pages 314-315]{saks71}.
\end{remark}

\begin{proof}
    \begin{enumerate}
        \item
            Start from the series expansion for the exponential function
            and then crudely estimate:
            \begin{align*}
              \EulerE^{n}
              &= 1 + \frac{n}{1!} + \cdots + \frac{n^{n-1}}{(n-1)!} + \\
              & \qquad  \frac{n^n}{n!}\left( 1 + \frac{n}{n+1} +
                \frac{n^2}{(n+1)(n+2)} + \dots \right) \\
                &< n \frac{n^n}{n!} + \frac{n^n}{n!} \sum_{j=0}^{\infty}
                \left( \frac{n}{n+1} \right)^j \\
                &= (2n+1) \frac{n^n}{n!}.
            \end{align*}
        \item
            On the other hand, \( \EulerE^n > n^n/n! \) by dropping all
            but the \( n^n/n! \) term from the series expansion for the
            exponential.
        \item
            Rearranging these two inequalities
            \[
                \frac{n^n}{\EulerE^n} < n!  < \frac{(2n+1) n^n}{\EulerE^n}.
            \]
        \item
            Now take the \( n \)th root of each term, and use o \( \sqrt
            [n]{2n+1} \to 1 \) as \( n \to \infty \).
    \end{enumerate}
\end{proof}

\subsection*{Some motivating functions}

The basic calculus fact
\[
    \lim_{x \to \infty} \left( 1 + \frac{1}{x} \right)^x = \EulerE
\] motivates studying the parametrized family of functions
\[
    a_{\alpha}(x) = \left( 1 + \frac{1}{x} \right)^{x + \alpha}.
\] Actually,
\[
    \lim_{x \to \infty} \left( 1 + \frac{1}{x} \right)^x = \EulerE
\] can be proved directly using information about the family of
functions, without resorting to L'Hospital's Rule.

Graphs of \( a_{\alpha}(x) \) for representative values of \( \alpha \)
are in Figure~%
\ref{fig:sequences:aalpha}.  The graph suggests \( a_{\alpha}(x) \) is
decreasing if \( \alpha > 1/2 \) and eventually increasing for \( \alpha
< 1/2 \).

\begin{figure}
    \centering
\begin{asy}
settings.outformat = "pdf";

import graph;

size(5inches);

real myfontsize = 12;
real mylineskip = 1.2*myfontsize;
pen mypen = fontsize(myfontsize, mylineskip);
defaultpen(mypen);

real e(real x) {
  return exp(1);
}

typedef real anon(real);
anon f(real alpha) {
  return new real(real x) {
    return ( 1 + 1/x)^(x + alpha);
  };
}

real[] a = {0.15, 0.35, 0.5, 0.75};
pen[] col = {red, orange, blue, black};
real M = 4;

for (int n=0; n < 4; ++n) {
  draw( graph( f(a[n]), 1, M), col[n]);
}

draw( graph(e, 1, M), p=green);

xaxis(L="$x$", xmin=0, xmax=M, RightTicks(format="%", N=9));
yaxis(L="$y$", ymin=0, ymax=4, LeftTicks(format="%", N=5));
\end{asy}
    \caption{Graphs of $ y = a_{\alpha}(x) $ for $ \alpha = 0.15 $,
    red, $ 0.35 $, orange, $ 0.5 $, blue, and $ 0.65 $, black.
    The green horizontal line is $ y = \EulerE $.}%
    \label{fig:sequences:aalpha}
\end{figure}

\begin{lemma}
    \label{lem:sequences:funcsalpha}
    \begin{enumerate}
        \item
            \( a_{\alpha}(x) \) decreases with \( x \) for \( x > 1\) and any
            fixed \( \alpha \ge 1/2 \)
        \item
            \( a_{\alpha}(x) \)  increases with \( x \) for \( x > \max[1,
            \frac{1}{3-6\alpha}] \) and any fixed \( \alpha < 1/2 \).
    \end{enumerate}
\end{lemma}

\begin{remark}
    The condition \( x > \max \left[ 1, \frac{1}{3-6\alpha} \right] \)
    reduces to \( x > 1 \) for \( \alpha \le \frac{1}{3} \) and \( x >
    1/(3-6\alpha) \) for \( \frac{1}{3} < \alpha < \frac{1}{2} \)
\end{remark}

\begin{remark}
    Essentially the same lemma is proved in a different way in Lemma~%
    \ref{lem:sequences:wallis}.
\end{remark}

\begin{remark}
    The results of this lemma are equivalent to a remark due to I.
    Schur, see
    \cite{polya98}, problem 168, on page 38 with solution on page 215,
    and a generalization appears in the \textit{College Mathematics
    Journal}, September 2020, Problem 1182, so this is frequently
    rediscovered.
\end{remark}

\begin{proof}
    \begin{enumerate}
        \item
            Introduce the parametrized family \( g_{\alpha}(x) = \log a_{\alpha}
            (x) = (x + \alpha) \log(1 + 1/x) \).
        \item
            Taking the derivative, rearranging and expanding in series:
            \begin{align*}
                g'(x) &= \log(1 + 1/x) + \frac{x+\alpha}{1 + 1/x} \cdot
                \frac{-1}{x^2} \\
                &= \left( \frac{1}{x^2 + x} \right) \left( (x^2 + x)
                \left( \frac{1}{x} - \frac{1/2}{x^2} + \frac{1/3}{x^3} -
                \cdots \right) - (x + \alpha) \right) \\
                &= \left( \frac{1}{x^2 + x} \right) \left( (\frac{1}{2}
                -\alpha) + \sum\limits_{k \ge 2} \frac{(-1)^{k-1}}{x^{k-1}}
                \right) \\
                &= \left( \frac{1}{x^2 + x} \right)\psi_c(x)
            \end{align*}
            where \( \psi_c(x) = \left( \frac{1}{2} - c \right) + \sum\limits_
            {k \ge 2} \frac{(-1)^{k-1}}{x^{k-1}} \).
        \item
            From the definitions
            \[
                \frac{\psi_{\alpha}(x)}{x^2 +x} = g'_{\alpha}(x) = \frac
                {a_{\alpha(x)}}{a_{\alpha}(x)}
            \] and \( x^2 + x > 0 \) and \( a_{\alpha}(x) > 0 \), so \(
            \psi_{\alpha}(x) \), \( g'_{\alpha}(x) \) and \( f_{\alpha}(x)
            \) all have the same sign. Therefore to prove the theorem it
            suffices to consider \( \psi_{\alpha}(x) \) in terms of \(
            \alpha \).
        \item
            The absolute ratio of consecutive terms in the alternating
            series in the definition of \( \psi_{\alpha}(x) \) is
            \[
                \frac{ \frac{1}{(k+1)(k+2)} \frac{1}{x^k}}{\frac{1}{k(k+1)}
                \frac{1}{x^{k-1}}} = \frac{k}{k+2} \left( \frac{1}{x}
                \right) < 1
            \] for \( k \ge 2 \) and \( x > 1 \).
        \item
            The alternating series has terms that decrease in magnitude
            and so
            \[
                \frac{1}{2} - \alpha - \frac{1}{6x} < \psi_{\alpha}(x) <
                \frac{1}{2} - \alpha
            \] for \( x > 1 \) and \( \alpha \ge 1/2 \).
        \item
            From the right inequality, \( \psi_{\alpha}(x) < \frac{1}{2}
            - \alpha \le 0 \) for \( \alpha \le \frac{1}{2} \) and \( x
            > 1 \).  This means \( a_{\alpha}'(x) < 0 \) for \( \alpha
            \le \frac{1}{2} \) and \( x > 1 \).  So \( a_{\alpha}(x) \)
            decreases for \( \alpha \le \frac{1}{2} \) and \( x > 1 \).
            This proves the first part of the lemma.
        \item
            When \( \alpha < \frac{1}{2} \) and \( x > 1 \) we have \(
            \frac{1}{2} - \alpha - \frac{1}{6x} \) which is equivalent
            to \( x(3 - 6 \alpha) \) which in turn is equivalent to \( x
            \ge \frac{1}{3 - 6c} \).
        \item
            Then from the left hand inequality,
            \[
                \psi_c(x) > \frac{1}{2} - \alpha - \frac{1}{6x} \ge 0
            \] for \( x > \max[1, \frac{1}{3-6\alpha}] \).
    \end{enumerate}
\end{proof}

\begin{corollary}
    \[
        \lim_{n \to \infty} \left( 1 + \frac{1}{n} \right)^n = \EulerE.
    \]
\end{corollary}

\begin{proof}
    \begin{enumerate}
        \item
            Write \( g_{\alpha}(x) \) as a power series as done for \(
            g'_{\alpha}(x) \):
            \begin{align*}
                g_{\alpha}(x) &= (x + \alpha) \log \left( 1 + \frac{1}{x}
                \right) \\
                &= (x + \alpha) \left(\frac{1}{x} - \frac{1/2}{x} + \frac{1/3}
                {x^3} - \cdots \right) \\
                &= 1 + \sum\limits_{k \ge 1} \left( \frac{k}{k+1} -c
                \right) \frac{(-1)^k}{k x^k}
            \end{align*}
        \item
            Now \( \lim_{x \to \infty} g_{\alpha}(x) = 1 \) for all \(
            \alpha \).
        \item
            Then
            \[
                \lim_{x \to \infty} \left( 1 + \frac{1}{x} \right)^x =
                \lim_{n \to \infty}\EulerE^{g_0(n)} = \EulerE.
            \]
    \end{enumerate}
\end{proof}

More is true, namely
\[
    \lim_{x \to \infty} \left( 1 + \frac{1}{x} \right)^{x + \alpha} =
    \lim_{n \to \infty}\EulerE^{g_\alpha(n)} = \EulerE
\] and what is more, the lemma says exactly how members of the
parametrized family approach \( \EulerE \),
\[
    \left( 1 + \frac{1}{n} \right)^{n + \alpha_1} < \EulerE < \left( 1 +
    \frac{1}{n} \right)^{n + \alpha_2}
\] for \( \alpha_1 < 1/2 \) and \( \alpha_2 \ge 1/2 \) and \( n \)
sufficiently large.

Now separately derive some equivalent formulations of these
inequalities.  First for \( \alpha_1 < 1/2 \), the following are
equivalent
\begin{align*}
    \EulerE &> \left( 1 + \frac{1}{n} \right)^{n + \alpha_1} \\
    \EulerE \left(\frac{n}{n+1} \right)^{n + \alpha_1} > 1\\
    \frac{\EulerE^{n+1}}{(n+1)^n (n+1)^\alpha_1} > \frac{\EulerE^n}{n^n
    n^\alpha_1} \\
    \frac{(n+1)!}{(n+1)^\alpha_1} \left( \frac{\EulerE}{n+1} \right)^{n+1}
    &> \frac{n!}{n^\alpha_1}\left( \frac{\EulerE}{n} \right)^n
\end{align*}

Similarly
\[
    \frac{(n+1)!}{(n+1)^\alpha_2} \left( \frac{\EulerE}{n+1} \right)^{n+1}
    < \frac{n!}{n^\alpha_2}\left( \frac{\EulerE}{n} \right)^n.
\]

Introducing the parametrized family of sequences
\[
    \frac{n}{n^\alpha}\left( \frac{\EulerE}{n} \right)^n,
\] the sequence increases with \( n \) for \( \alpha < 1/2 \) and \( n >
\max[1, 1/(3-6 \alpha)] \) and the sequence decreases with \( n \) for \(
\alpha \ge 1/2 \) and \( n > 1 \).  For fixed \( n \), the sequence is a
decreasing function of \( \alpha \).

By monotonicity, the sequences
\[
    \frac{n!}{n^\alpha}\left( \frac{\EulerE}{n} \right)^n,
\] must approach \( 0 \), \( \infty \), or some finite \( L \) as \( n
\to \infty \).  If \( \alpha < 1/2 \), the sequence is increasing and so
the limit must be either \( \infty \) or \( L \).  If some \( \alpha_0 <
1/2 \) the limit is \( L < \infty \), then for fixed \( \beta \in (\alpha_0,
1/2) \),
\begin{multline*}
    \lim_{n \to \infty}\frac{n!}{n^\beta}\left( \frac{\EulerE}{n} \right)^n
    = \lim_{n \to \infty} \frac{n^{\alpha_0}}{n^{\beta}}\frac{n!}{n^\alpha_0}\left
    ( \frac{\EulerE}{n} \right)^n = \\
    \lim_{n \to \infty} \frac{n^{\alpha_0}}{n^{\beta}} \cdot \lim_{n \to
    \infty}\frac{n!}{n^\alpha_0}\left( \frac{\EulerE}{n} \right)^n = 0
    \cdot L = 0
\end{multline*}
which is not possible, so the limit must be \( \infty \) for all \(
\alpha < 1/2 \).

Similarly, if \( \alpha > 1/2 \), the sequence is decreasing to a limit
of \( 0 \), see the exercises.

This leaves just \( \alpha =1/2 \) the only possible member of the
family of sequences that could approach a finite limit as \( n \to
\infty \). The goal is now to show this finite limit exists.

\subsection*{Stirling's Formula from sequences and series}

Rearranging slightly, let \( a_n = \frac{n!} {n^{n+1/2}\EulerE^{-n}} \)\,.
Then \( \frac{a_n}{a_{n+1}} = \left(\frac{n+1}{n} \right)^{n + 1/2}\EulerE^
{-1} \) and \( \log \left( \frac{a_n}{a_{n+1}} \right) = (n+1/2) \log( 1
+ 1/n) - 1 \)\,.

\begin{lemma}
    For \( |x| < 1 \),
    \[
        \log\left( \frac{1+x}{1-x} \right) = 2 \sum_{k=0}^{\infty} \frac
        {1}{(2k+1)x^{2k+1} }\,.
    \]%
    \label{lem:sequences:lemma1}
\end{lemma}

\begin{proof}
    Left as an exercise.
\end{proof}

Note then that
\[
    \log\left( 1 + \frac{1}{n} \right) = \log\left( \frac{ 1 + \frac{1}{2n+1}}
    {1 - \frac{1}{2n+1}} \right) = 2 \sum_{k=0}^{\infty} \frac{1}{(2k+1)
    (2n+1)^{2k+1}}\,.
\] Then
\begin{align*}
    \log \left(\frac{a_n}{a_{n+1}} \right) &= \left(\frac{2n+1}{2}
    \right) \left(2 \sum_{k=0}^{\infty} \frac{1}{(2k+1)(2n+1)^{2k+1}}
    \right) -1 \\
    &= \sum_{k=1}^{\infty} \frac{1}{(2k+1)(2n+1)^{2k}} = \sum_{k=0}^{\infty}
    \frac{1}{(2k+3)(2n+1)^{2k+2}}\,.
\end{align*}
Now coarsely estimating the denominators
\[
    \log \left(\frac{a_n}{a_{n+1}} \right) \le \frac{1}{2(2n+1)^2} \sum_
    {k=0}^ {\infty} \frac{1}{(2n+1)^{2k}}\,.
\]

\begin{lemma}
    \[
        \log(a_{n+1} ) < \log \left(a_n \right) < \frac{1}{12n} - \frac{1}
        {12(n+1)} + \log(a_{n+1} ).
    \]%
    \label{lem:sequences:lemma2}
\end{lemma}

\begin{proof}
    \begin{enumerate}
        \item
            Let \( f(x) = (x + 1/2)\log(1 + 1/x) - 1 \), so \( \log
            \left( \frac {a_n}{a_{n+1}} \right) = f(n) \).
        \item
            An exercise proves \( f(x) \to 0 \) as \( x \to \infty \).
        \item
            Also \( f'(x) = \log(1 + 1/x) - \frac{2x + 1}{2x^2 + 2x} \).
        \item
            Because \( f'(x) < 0 \) for \( x > 1 \) (proof left as an
            exercise) \( f(x) \) is decreasing from \( (3/2)\log(2) - 1
            \approx 0.0397 \) to \( 0 \) as \( x \) increases.
        \item
            Hence \( 0 < \log \left(\frac {a_n}{a_{n+1}} \right) \).
        \item
            By Lemma~%
            \ref{lem:sequences:lemma1}
            \[
                \log \left(\frac{a_n}{a_{n+1}} \right) \le \frac{1}{3(2n+1)^2}
                \sum_{k=0}^{\infty} \frac{1}{(2n+1)^{2k}}\,.
            \]
        \item
            The sum is a geometric sum, so
            \[
                0 < \log \left(\frac{a_n}{a_{n+1}} \right) < \frac{1}{3(2n+1)^2}
                \frac{1}{1 - (1/(2n+1)^2)} = \frac{1}{12n(n+1)}.
            \]
        \item
            Expand \( 1/(12n(n+1)) \) in partial fractions and add \(
            \log(a_{n+1}) \) throughout to get
            \[
                \log(a_{n+1} ) < \log(a_n ) < \frac{1}{12n} - \frac{1}{12
                (n+1)} + \log(a_{n+1} ).
            \]
    \end{enumerate}
\end{proof}

Define \( x_n = \log(a_{n} ) - \frac{1}{12n} \), so Lemma~%
\ref{lem:sequences:lemma2} shows that \( x_n \) is an increasing
sequence
\begin{equation}
    x_n = \log(a_n) -\frac{1}{12n} < \log(a_{n+1}) - \frac{1}{12(n+1)} =
    x_{n+1}.%
    \label{eq:sequences:inequality}
\end{equation}
Define \( y_n = \log(a_n) \), and then the left-side inequality in
Lemma~%
\ref{lem:sequences:lemma2} shows that \( y_{n} \) is a decreasing
sequence that is, \( y_{n+1} < y_n \).  By the definition of \( x_n \)
and \( y_n \), \( x_n < y_n \) and \( |x_n - y_n| = 1/(12n) \), so \( |x_n
- y_n| \to 0 \) as \( n \to \infty \).  Therefore \( \sup x_{n}= \inf y_
{n} \) and call the common value \( \lambda \).  By continuity, \( \lim_
{n \to \infty} a_{n} = \EulerE^\lambda \).

Using elementary properties of limits
\begin{equation}
    \EulerE^\lambda = \lim_{n \to \infty} a_{n} = \frac{\left(\lim_{n
    \to \infty} a_{n} \right)^2 }{\lim_{n \to \infty} a_{2n} } = \lim_{n
    \to \infty} \frac{\left(a_{n} \right)^2}{a_{2n}}\,.%
    \label{eq:sequences:limit}
\end{equation}
However,
\begin{equation}
    \frac{\left(a_{n} \right)^2}{a_{2n}} = \sqrt{\left(\frac{2}{n}
    \right) } \frac{2 \cdot 4 \dots (2n-2) \cdot 2n}{1 \cdot 3 \dots (2n-3)
    \cdot (2n-1)} \,.%
    \label{eq:sequences:wallis}
\end{equation}
The demonstration is left as an exercise.

\begin{lemma}[Wallis' Formula]
    \[
        \lim_{n \to \infty}\left(\frac{(2n)\cdot(2n)\dots 2 \cdot 2}{(2n+1)
        \cdot (2n-1) \cdot (2n-1) \dots 3 \cdot 3 \cdot 1} \right) =
        \frac{\PI}{2}.
    \]
\end{lemma}
\index{Wallis Formula}

\begin{proof}
    See the proofs in \link{../WallisFormula/wallisformula.xml}{Wallis
    Formula}.
\end{proof}

Using the continuity of the square root function
\[
    \lim_{n \to \infty}\left(\sqrt{ \frac{1}{2n+1} } \right) \left(\frac
    {(2n)\cdot(2n-2)\dots 4 \cdot 2}{ (2n-1) \cdot (2n-3) \dots 5 \cdot
    3 \cdot 1} \right) = \sqrt{ \frac{\PI}{2} }\,.
\]

Now multiplying both sides by \( 2 \) and rewriting the leading square
root sequence, get
\[
    \lim_{n \to \infty}\left(\sqrt{ \frac{2}{n} \, \frac{2n}{2n+1} }
    \right) \left(\frac{(2n)\cdot(2n-2)\dots 4 \cdot 2}{ (2n-1) \cdot (2n-3)
    \dots 5 \cdot 3 \cdot 1} \right) = \sqrt{ 2 \PI }\,.
\] Then since
\[
    \lim_{n \to \infty} \sqrt{ \frac{2n}{2n+1} } = 1,
\] equation~%
\ref{eq:sequences:wallis} is
\[
    \EulerE^\lambda = \lim_{n \to \infty} a_{n} = \sqrt{ \left(\frac{2}{n}
    \right) } \frac{ 2 \cdot 4 \dots (2n-2) \cdot 2n }{ 1 \cdot 3 \dots
    (2n-3) \cdot (2n-1) } = \sqrt{ 2 \PI } \,.
\] Equivalently, unwrapping the definition of \( a_{n} = \frac{n!}{n^{n+1/2}\EulerE^
{-n}} \) this is exactly Stirling's Formula
\[
    n!  \asympt \sqrt{2 \PI} n^{n+1/2} \EulerE^{-n}\,.
\]

Using the definitions \( x_n = \log(a_{n} ) - \frac{1}{12n} \) and \( y_n
= \log(a_n) \), the inequality \( x_n < y_n \), and the least upper
bound and greatest lower bound limit in equation~(%
\ref{eq:sequences:limit}) we can express Stirling's Formula in
inequality form
\[
    \sqrt{2 \PI} n^{n+1/2} \EulerE^{-n} < n!  < \sqrt{2 \PI} n^{n+1/2}
    \EulerE^{-n + 1/(12n)}\,.
\] This is almost as good as the inequality~(%
\ref{eq:sequences:stirlingsinequality}).  This also gives a proof of the
simple and useful inequality about Stirling's Formula, for all \( n \)
\[
    \sqrt{2 \PI} n^{n+1/2} \EulerE^{-n} < n!\,.
\]

\begin{remark}
    This proof of Stirling's Formula and the inequality~(%
    \ref{eq:sequences:stirlingsinequality}) is the easiest, the shortest
    and the most elementary of the Stirling's Formula proofs.  These are
    all definite advantages.  The main disadvantage of this proof is
    that it requires the form of Stirling's Formula.  However, this form
    follows from investigation of a natural family of functions
    motivated by the definition of Euler's constant, the base of the
    natural logarithms.
\end{remark}

\subsection*{Stirling's Formula from Wallis' Formula}

\begin{lemma}
    \label{lem:sequences:wallis} For \( \alpha \in \Reals \), the
    sequence
    \[
        a_{\alpha}(n) = \left( 1 + \frac{1}{n} \right)^{n + \alpha}
    \] is decreasing if \( \alpha \in [ \frac{1}{2}, \infty ) \), and
    increasing for \( n \ge N(\alpha) \) if \( \alpha \in (-\infty,
    \frac{1}{2} ) \).
\end{lemma}

\begin{remark}
    An illustration of representative sequences with \( \alpha = 1 \) (green)
    and \( \alpha = 0 \) (red) is in Figure~%
    \ref{fig:sequences:lemma}.
\end{remark}

\begin{figure}
  \centering
  \begin{asy}
settings.outformat = "pdf";

import graph;

size(5inches);

real myfontsize = 12;
real mylineskip = 1.2*myfontsize;
pen mypen = fontsize(myfontsize, mylineskip);
defaultpen(mypen);

real M = 10;

for (int i=1; i<11; ++i) {
  dot( (i, (1 + 1/i)^(i + 1 ) ), green);
  dot( (i, (1 + 1/i)^(i) ), red);
}

xaxis(L="$x$", xmin=0, xmax=M, RightTicks(format="%", N=10));
yaxis(L="$y$", ymin=0, ymax=4, LeftTicks(format="%", N=4));
\end{asy}
    %% \includegraphics{lemma_illustration}
    \caption{Example of the lemma with $ \alpha = 1 $ (green) and $
    \alpha = 0 $ (red).  }%
    \label{fig:sequences:lemma}
\end{figure}

\begin{remark}
    This lemma is based on a remark due to I. Schur, see
    \cite{polya98}, problem 168, on page 38 with solution on page 215.
\end{remark}

\begin{proof}
    \begin{enumerate}
        \item
            The derivative of the function \( f(x) = \left( 1 + \frac{1}
            {x} \right)^{x + \alpha} \) (defined on \( [1, \infty) \))
            is
            \[
                f'(x) = \left( 1 + \frac{1}{x} \right)^{x + \alpha}
                \left( \log\left( 1 + \frac{1}{x} \right) - \frac{x+\alpha}
                {x(x+1)} \right).
            \]
        \item
            Let
            \[
                g(x) = \left( \log\left( 1 + \frac{1}{x} \right) - \frac{x+\alpha}
                {x(x+1)} \right)
            \] then
            \[
                g'(x) = \frac{(2\alpha -1) x + \alpha}{x^2(x+1)^2}
            \] and \( \lim_{x \to \infty} g(x) = 0 \).
        \item
            It follows that \( g(x) < 0 \) and so \( f'(x) < 0 \) when \(
            \alpha \ge 1/2 \) and \( x \ge 1 \), and \( f'(x) > 0 \)
            when \( \alpha < 1/2 \) and \( x \ge \max(1, \frac{\alpha}{1
            - 2\alpha} ) \).  The monotonicity of \( a_{\alpha}(n) \)
            follows.
    \end{enumerate}
\end{proof}

From the lemma, for every \( \alpha \in (0, 1/2) \) there is a positive
integer \( N(\alpha) \) such that
\[
    \left( 1 + \frac{1}{k} \right)^{k + \alpha} < \EulerE < \left( 1 +
    \frac{1}{k} \right)^{k + 1/2}
\] for all \( k \ge N(\alpha) \).  As a consequence, we get
\[
    \prod_{k=n}^{2n-1} \left( 1 + \frac{1}{k} \right)^{k + \alpha} <
    \EulerE^n < \prod_{k=n}^{2n-1} \left( 1 + \frac{1}{k} \right)^{k +
    1/2}.
\]

Rearrange the products with telescoping cancellations, using the upper
bound on the right as an example.
\begin{align*}
    & \prod_{k=n}^{2n-1} \left( 1 + \frac{1}{k} \right)^{k + 1/2} \\
    &\qquad = \left( 1 + \frac{1}{n} \right)^{n + 1/2} \left( 1 + \frac{1}
    {n+1} \right)^{n+1 + 1/2} \cdots \left( 1 + \frac{1}{2n-1} \right)^{2n-1
    + 1/2} \\
    &\qquad = \left(\frac{n+1}{n} \right)^{n + 1/2} \left(\frac{n+2}{n+1}
    \right)^{n+1 + 1/2} \cdots \left(\frac{2n}{2n-1} \right)^{2n-1 + 1/2}
    \\
    &\qquad = \left(\frac{n+1}{n} \right)^{1/2} \left(\frac{n+2}{n+1}
      \right)^{1/2} \cdots \left(\frac{2n}{2n-1} \right)^{1/2} \cdot \\
    & \qquad \qquad \left
    ( \frac{n+1}{n} \right)^{n} \left(\frac{n+2}{n+1} \right)^{n+1}
    \cdots \left(\frac{2n}{2n-1} \right)^{2n-1} \\
    &\qquad = \left( \frac{2n}{n} \right)^{1/2} \cdot \left( \frac{ (2n)^
    {2n-1}}{n^n \cdot (n+1) \cdots (2n-1)} \right) \\
    &\qquad = 2^{1/2} \cdot \frac{ (2n)^{2n-1}}{n^n \cdot (n+1) \cdots (2n-1)}
    \\
    &\qquad = 2^{1/2} \cdot \frac{ 2^{2n-1} \cdot n^{n-1}}{ (n+1) \cdots
    (2n-1)}.
\end{align*}
Multiply by the last fraction by \( n!/n^n \) and Write more compactly,
\begin{align*}
    & \frac{ n!}{ n^n } \cdot 2^{1/2} \cdot \left( \frac{ 2^{2n-1} \cdot
    n^{n-1}}{ (n+1) \cdots (2n-1)} \right).  \\
    & \qquad = 2^{1/2} \cdot \frac{ 2^{n-1} \cdot 2^n \cdot n!}{n \cdot
    (n+1) \cdots (2n-1) } \\
    & \qquad = 2^{1/2} \cdot \frac{(2n)!!}{(2n-1)!!}.
\end{align*}
This uses the \defn{double factorial} notation \( n!!  = n \cdot (n-2)
\cdots 4 \cdot 2 \) if \( n \) is even, and \( n!!  = n \cdot (n-2)
\cdots 3 \cdot 1 \) if \( n \) is odd.%
\index{double factorial}
The lower bound product on the left is similar.

After multiplying through by \( n!/n^{n+1/2} \),
\[
    \frac{2^\alpha}{\sqrt{n}} \cdot \frac{ (2n)!!}{(2n-1)!!} < \frac{n!
    \EulerE^n}{n^{n+1/2}} < \frac{2^{1/2}}{\sqrt{n}} \cdot \frac{(2n)!!}
    {(2n-1)!!}
\] for all \( n \ge N(\alpha) \).  Using the Wallis Formula,%
\index{Wallis Formula}
\[
    2^\alpha \sqrt{\PI} \le \liminf_{n \to \infty} \frac{n!  \EulerE^n}{n^
    {n+1/2}} \le \limsup_{n \to \infty} \frac{n!  \EulerE^n}{n^{n+1/2}}
    \le \sqrt{2\PI}.
\] Stirling's formula follows by passing to the limit as \( \alpha \to
1/2 \).

\begin{remark}
    This proof generalizes to an asymptotic formula for the Gamma
    function using log-convexity of the Gamma function.  See
    \cite{dutkay13}.
\end{remark}

\subsection*{Related Asymptotic Formulas for Binomial Coefficients}

\begin{theorem}
    \[
        \left( \frac{n}{k} \right)^k \le \binom{n}{k} \le \left( \frac{\EulerE
        n}{k} \right)^k
    \]
\end{theorem}
\index{binomial coefficients!inequalities}

\begin{proof}
    \begin{enumerate}
        \item
            Start with
            \[
                \binom{n}{k} = \frac{n(n-1) \cdots (n-k+2)(n-k+1)}{k(k-1)\cdots
                2 \cdot 1}.
            \]
        \item
            Since
            \[
                \frac{n}{k} \le \frac{n-i}{k-i}
            \] for all \( i = 0, 1, \dots k-1 \), the left inequality is
            immediate.
        \item
            Since
            \[
                \binom{n}{k} = \frac{n(n-1) \cdots (n-k+2)(n-k+1)}{k(k-1)\cdots
                2 \cdot 1} \le \frac{n^k}{k!}
            \] and by a step in the proof of Theorem~%
            \ref{thm:sequences:weakform}
            \[
                \frac{1}{k!} \le \left( \frac{e}{k} \right)^{k}
            \] the right inequality follows.
    \end{enumerate}
\end{proof}

\begin{theorem}
    \[
        \binom{n}{k} \asympt \left(\frac{n^k}{k!} \right)
    \] if \( k = o(n^{1/2}) \) as \( n \to \infty \).
\end{theorem}
\index{binomial coefficients!asymptotics}

\begin{proof}
    \begin{enumerate}
        \item
            The statement of the theorem is equivalent to showing
            \[
                \lim_{n \to \infty}\frac{n(n-1) \cdots (n-k+1)}{n^k} =
                \lim_{n \to \infty} \prod_{j=1}^{k-1}\left( 1 - j/n
                \right) = 1
            \] if \( k = o(n^{1/2}) \).
        \item
            In turn, this is equivalent to showing
            \[
                \lim_{n \to \infty} \log \left( \prod_{j=1}^{k-1} \left(
                1 - j/n \right) \right) = \lim_{n \to \infty} \sum_{j=1}^
                {k-1} \log \left( 1 - j/n \right) = 0.
            \]
        \item
            Using \( \log(1-x) = x + O(x^2) \), the sum is
            \[
                \sum_{j=1}^{k-1} \log \left( 1 - j/n \right) = \sum_{j=1}^
                {k-1} \left( j/n + O(j^2/n^2) \right) = \frac{k(k-1)}{n}
                + \frac{1}{n^2} O(k^3).
            \]
        \item
            Now the hypothesis \( k = o(n^{1/2}) \) comes into play so \(
            \lim_{n \to \infty }\frac{k(k-1)}{n} = 0 \) and \( \lim_{n
            \to \infty } \frac{1}{n^2} O(k^3) = 0 \) establishing the
            desired limit.
    \end{enumerate}
\end{proof}

\visual{Section Starter Question}{../../../../CommonInformation/Lessons/question_mark.png}
\section*{Section Ending Answer}

The geometric summation formula is
\[
    \frac{1}{1-x} = \sum\limits_{k=0}^{\infty} x^k
\] which converges uniformly for \( \abs{x} < 1 \).  Then
\[
    \frac{1}{1+x} = \sum\limits_{k=0}^{\infty} (-1)^{k}x^k
\] and
\[
    \log(1 + x) = \int\limits_{0}^{x} \frac{1}{1 + t} \df{t} = \sum\limits_
    {k=0}^{\infty} (-1)^k \frac{x^{k+1}}{k+1}
\] where the term-by-term integration is justified by the uniform
convergence for \( \abs{x} < 1 \).

\subsection*{Sources}

The weak form of Stirling's Formula is taken from
\cite{saks71}.  The motivating derivation of the sequences from the
family of functions is from
\cite{hammett20}.  The first sequence proof is adapted from the sketch
in Kazarinoff
\cite{kazarinoff61} based on the note by Nanjundiah
\cite{nanjundiah59}.  The second sequence proof using derivatives and
monotonicity is adapted from
\cite{dutkay13}.  The binomial coefficient limits are from lecture notes
by Xavier Perez Gimenez.

\hr

\visual{Problems to Work}{../../../../CommonInformation/Lessons/solveproblems.png}
\section*{Problems to Work for Understanding}

\renewcommand{\theexerciseseries}{}
\renewcommand{\theexercise}{\arabic{exercise}}

\begin{exercise}
    Show \( \sqrt[n]{2n+1} \to 1 \) as \( n \to \infty \).
\end{exercise}
\begin{solution}
    Take logarithms and consider \( \log(2n+1)/n \) which
    approaches \( 0 \) as \( n \to 0 \).  Then \( \sqrt[n]{2n+1}
    \to \EulerE^0 = 1 \) as \( n \to \infty \).
\end{solution}
\begin{exercise}
    Show for \( |x| < 1 \),
    \[
        \log\left(\frac{1+x}{1-x} \right) = 2 \sum_{k=0}^{\infty}
        \frac{1}{(2k+1)x^{2k+1} }\,.
    \]
\end{exercise}
\begin{solution}
    \begin{align*}
        \log\left(\frac{1+x}{1-x} \right) &= \log(1+x) - \log(1-x)\\
        = \sum_{k=0}^{\infty}\frac{(-1)^k x^{k+1} }{k+1} - \sum_
        {k=0}^{\infty}\frac{x^{k+1}}{(2k+1) }\\
        = -2 \sum{k=1,3,5,\dots} \frac{x^{k+1}}{k }\\
        = -2 \sum_{k=0}^{\infty} \frac{x^{2k+1}}{(2k+1) }\,.
    \end{align*}
\end{solution}
\begin{exercise}
Show that if \( \alpha > 1/2 \)
\[
    \lim_{n \to \infty}\frac{n!}{n^\alpha}\left( \frac{\EulerE}{n}
    \right)^n = 0.
  \]
\end{exercise}      
\begin{solution}
    If \( \alpha > 1/2 \), the sequence is positive and
    decreasing and so the limit must be either \( 0 \) or \( L
    >0 \).  If for some \( \alpha_0 > 1/2 \) the limit is \( L <
    0 \), then for fixed \( \beta \in (1/2, \alpha_0) \),
    \begin{multline*}
        \lim_{n \to \infty}\frac{n!}{n^\beta}\left( \frac{\EulerE}
        {n} \right)^n = \lim_{n \to \infty} \frac{n^{\alpha_0}}{n^
        {\beta}}\frac{n!}{n^\alpha_0}\left( \frac{\EulerE}{n}
        \right)^n = \\
        \lim_{n \to \infty} \frac{n^{\alpha_0}}{n^{\beta}} \cdot
        \lim_{n \to \infty}\frac{n!}{n^\alpha_0}\left( \frac{\EulerE}
        {n} \right)^n = \infty \cdot L = \infty
    \end{multline*}
    which is not possible, so the limit must be \( 0 \) for all \(
    \alpha > 1/2 \).
\end{solution}
\begin{exercise}
    Let \( f(x) = (x + 1/2)\log(1 + 1/x) - 1 \).  Show \( f(x) \)
    decreases to \( 0 \) as \( x \to \infty \).
\end{exercise}
\begin{solution}
    This is essentially the content of steps \( 1 \) through \(
    6 \) of Lemma~%
    \ref{lem:sequences:funcsalpha}.  Using the notation in the
    proof in step \( 6 \)
    \[
        f'(x) = a_{1/2}'(x) = - \frac{1}{6x} < \psi_{\alpha}(x)
        < 0.
    \] Using, for instance L'Hospital's rule, \( \lim_{x \to
    \infty}(x + 1/2)\log(1 + 1/x) = 1 \), or expanding as a
    power series in terms of \( 1/x \)
    \[
        (x + 1/2)\log(1 + 1/x) = 1 + \frac{1}{12 y^2} - \frac{1}
        {12 y^3} + \dots
    \] so \( \lim_{x \to \infty}(x + 1/2)\log(1 + 1/x) = 1 \).
\end{solution}
\begin{exercise}
    Show
    \[
        \frac{ \left(a_{n} \right)^2 }{ a_{2n} } = \sqrt{ \left(\frac
        {2}{n} \right) } \frac{ 2 \cdot 4 \dots (2n-2) \cdot 2n
        }{ 1 \cdot 3 \dots (2n-3) \cdot (2n-1) } \,.
    \]
\end{exercise}
\begin{solution}
    Recall \( a_n = \frac{n!} {n^{n+1/2}\EulerE^{-n}} \) so \( (a_n)^2
    = \frac{n!  \cdot n!} {n^{2n+1}\EulerE^{-2n}} \).  Also \( a_
    {2n} = \frac{(2n)!} {(2n)^{2n+1/2}\EulerE^{-2n}} \). Then
    \begin{align*}
        \frac{ \left(a_{n} \right)^2 }{ a_{2n} } &= \left( \frac
        {n! \cdot n!}{(2n)!} \right) \frac{2^{2n} 2^{1/2} n^{2n}
        n^{1/2}}{n^{2n} n} \\
        &= \sqrt{\frac{2}{n}} \left( \frac{n!\cdot n!}{(2n)!}
        \right) 2^{2n} \\
        &= \sqrt{\frac{2}{n}} \left( \frac{n!}{1 \cdot 3 \cdots (2n-3)
        \cdot (2n-1)} 2^{n} \right)\\
        &= \sqrt{\left(\frac{2}{n} \right) } \frac{2 \cdot 4
        \cdots (2n-2) \cdot (2n)}{1 \cdot 3 \cdots (2n-3) \cdot
        (2n-1)} \,.
    \end{align*}
\end{solution}
\begin{exercise}
    Show the derivative of the function \( f(x) = \left( 1 +
    \frac{1}{x} \right) ^{x + \alpha} \) (defined on \( [1,
    \infty) \)) is
    \[
        f'(x) = \left( 1 + \frac{1}{x} \right)^{x + \alpha}
        \left( \log\left( 1 + \frac{1}{x} \right) - \frac{x+\alpha}
        {x(x+1)} \right).
    \]
\end{exercise}
\begin{solution}
    Consider \( \log(f(x)) = (x + \alpha) \log(1 + 1/x) \) so
    \begin{align*}
        \left[ \log(f(x)) \right]' &= \frac{f'(x)}{f(x)} \\
        &= \log\left( 1 + \frac{1}{x} \right) + (x+\alpha) \cdot
        \frac{x}{x+1} \cdot \frac{-1}{x^2} \\
        & \log\left( 1 + \frac{1}{x} \right) _{} (x+\alpha) \cdot
        \frac{-1}{x(x+1)}.
    \end{align*}
    Then
    \[
        f'(x) = \left( 1 + \frac{1}{x} \right)^{x + \alpha}
        \left( \log\left( 1 + \frac{1}{x} \right) - \frac{x+\alpha}
        {x(x+1)} \right).
    \]
\end{solution}

\hr

\visual{Books}{../../../../CommonInformation/Lessons/books.png}
\section*{Reading Suggestion:}

\bibliography{../../../../CommonInformation/bibliography}

%   \begin{enumerate}
%     \item
%     \item
%     \item
%   \end{enumerate}

\hr

\visual{Links}{../../../../CommonInformation/Lessons/chainlink.png}
\section*{Outside Readings and Links:} % \begin{enumerate}
%   \item
%   \item
%   \item
%   \item
% \end{enumerate}

\section*{\solutionsname}
\loadSolutions

\hr

\mydisclaim \myfooter

Last modified:  \flastmod

\end{document}

File name                  : sequences_new.tex
Number of characters       : 35425
Number of words            : 3277
Percent of complex words   : 15.01
Average syllables per word : 1.6729
Number of sentences        : 83
Average words per sentence : 39.4819
Number of text lines       : 817
Number of blank lines      : 207
Number of paragraphs       : 197


READABILITY INDICES

Fog                        : 21.7983
Flesch                     : 25.2359
Flesch-Kincaid             : 19.5478





%%% Local Variables:
%%% TeX-master: t
%%% End:
